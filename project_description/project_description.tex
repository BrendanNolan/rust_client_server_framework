\documentclass{beamer}
\mode<presentation>
{
  \usetheme{NYU}      % or try Darmstadt, Madrid, Warsaw, ...
  \usecolortheme{default} % or try albatross, beaver, crane, ...
  \usefonttheme{default}  % or try serif, structurebold, ...
  \setbeamertemplate{navigation symbols}{}
  \setbeamertemplate{caption}[numbered]
} 

\usepackage[english]{babel}
\usepackage[T1]{fontenc}
\usepackage[utf8x]{inputenc}
\usepackage{hyperref}
\usepackage{listings}
\usepackage{mdframed}
\usepackage{xcolor}
\usepackage{graphicx}

\definecolor{light-gray}{gray}{0.95}

\title[]{Generic Application-Server Crates}
\subtitle{Built With Async Rust}
\author{Brendan Nolan}
\date{\today}

\begin{document}

\begin{frame}
  \titlepage
\end{frame}

\begin{frame}{Goal}
  \begin{itemize}[<+->]
    \item<1> Provide an \emph{application-server} crate and a \emph{client} crate, allowing users to request jobs which are
          too heavy for their local machines.
    \item<2> Example use case: scientists in a wetlab, who must regularly run heavy models from their bench
          laptops.
  \end{itemize}
\end{frame}

\begin{frame}[fragile]{How A User Creates A Custom Server}
  A user who wants a server that provides weather status must only do the following:
  \begin{mdframed}[backgroundcolor=light-gray, roundcorner=10pt,leftmargin=1, rightmargin=1, innerleftmargin=1, innerrightmargin=1, innertopmargin=1,innerbottommargin=1, outerlinewidth=1, linecolor=light-gray]
    \begin{lstlisting}[basicstyle=\tiny,]
let (tx_shutdown, rx_shutdown) = watch::channel(());
let server_task = tokio::spawn(server_runner::run_server(
    "<IP Address>:<Port>",
    weather_provider,  // User's type - must implement RequestProcessor
    ShutdownListener::new(rx_shutdown),
));
    \end{lstlisting}
  \end{mdframed}
\end{frame}


\begin{frame}[fragile]{Example Application}
  \begin{figure}[htp]
    \centering
    \includegraphics[width=11cm, height=5cm]{weather_tui}
  \end{figure}
\end{frame}

\begin{frame}{Must-Have Features}
  For crate users
  \begin{itemize}[<+->]
    \item<1> Easily customise server data processing, e.g.:
    \begin{itemize}[<+->]
      \item<1> train ML algorithms
      \item<1> run planetary models
      \item<1> run structural analysis of timberframe houses
    \end{itemize}
    \item<2> Allow client to choose between
      \begin{itemize}[<+->]
        \item<2> call-and-response model 
        \item<2> sending and receiving in any order
      \end{itemize}
    \item<3> Remain responsive as number of clients connecting to a single server grows
  \end{itemize}
\end{frame}

\begin{frame}{Must-Have Features}
  For crate developers
  \begin{itemize}[<+->]
    \item<1> Do not waste resources by blocking threads while waiting for I/O
    \item<2> (More insidious than the above) do not make asynchronous work wait for other asynchronous work
      unnecessarily
  \end{itemize}
\end{frame}

\begin{frame}{Technology Choices}
  Criteria for chosen technology
  \begin{itemize}[<+->]
    \item<1> Has well documented library(s) for async I/O
    \item<2> Eliminates as many bugs as possible as early as possible (strong bias towards compiled languages)
  \end{itemize}
\end{frame}

\begin{frame}{Technology Choices}
  \emph{C++}: \emph{boost.asio}
  \begin{itemize}[<+->]
    \item<1> Pros
    \begin{itemize}[<+->]
      \item<1> \emph{boost.asio} is a mature library
    \end{itemize}
    \item<2> Cons
    \begin{itemize}[<+->]
      \item<2> awkward call-back mechanism
      \item<2> many object-lifetime footguns (use-after-free bugs)
      \item<2> static polymoprphism extremely awkward to use
    \end{itemize}
  \end{itemize}
\end{frame}

\begin{frame}{Technology Choices}
  \emph{Rust}: \emph{tokio}
  \begin{itemize}[<+->]
    \item<1> Cons
      \begin{itemize}[<+->]
        \item<1> async I/O ecosystem in \emph{Rust} is still young
      \end{itemize}
    \item<2> Pros
    \begin{itemize}[<+->]
      \item<2> \emph{async}/\emph{await} syntax is far more ergonomic than passing callbacks around
      \item<2> typesystem eliminates use-after-free bugs
      \item<2> static polymorphism is user friendly
    \end{itemize}
  \end{itemize}
\end{frame}

\begin{frame}{Winner}
  \begin{large}
    \begin{center}
      \textbf{Winner: \emph{Rust} with \emph{tokio}}
    \end{center}
  \end{large}
\end{frame}

\begin{frame}[fragile]{Design Decisions}
  \begin{itemize}[<+->]
    \item<1> Customization point should allow state (e.g. in order to cache results of previous model runs)
  \end{itemize}
  \begin{mdframed}[backgroundcolor=light-gray, roundcorner=10pt,leftmargin=1, rightmargin=1, innerleftmargin=1, innerrightmargin=1, innertopmargin=1,innerbottommargin=1, outerlinewidth=1, linecolor=light-gray]
    \begin{lstlisting}[basicstyle=\tiny,]
pub trait RequestProcessor<Req, Resp> {
    fn process(&self, request: &Req) -> Resp;
}
    \end{lstlisting}
  \end{mdframed}
\end{frame}

\begin{frame}{Design Decisions}
  \begin{itemize}[<+->]
    \item<1> Support serialization by \emph{serde} only
    \item<2> Make three crates (for client, server, and I/O utilities) rather than combining all into one crate
  \end{itemize}
\end{frame}

\begin{frame}{Design Decisions}
  \begin{itemize}
    \item<1> Follow the Golang philosophy: "Do not communicate by sharing memory - share memory by communicating"
    \item<2> Employ the \emph{actor} pattern, recommended by Alice Ryhl (tokio maintainer)
    \item<3> Make it as simple as possible to set up a custom server
  \end{itemize}
\end{frame}

\begin{frame}[fragile]{How The Server Handles Jobs}
  My Implementation Of The Actor Pattern \newline
  \begin{tiny}
    (Some generic parameters and trait bounds removed for readability)
  \end{tiny}
  \begin{mdframed}[backgroundcolor=light-gray, roundcorner=10pt,leftmargin=1, rightmargin=1, innerleftmargin=1, innerrightmargin=1, innertopmargin=1,innerbottommargin=1, outerlinewidth=1, linecolor=light-gray]
    \begin{lstlisting}[basicstyle=\tiny,]
pub struct Command<Req, Resp> {
    pub data: Req,
    pub responder: oneshot::Sender<Resp>,
}

// The "Handle"
pub struct JobDispatcher<Req, Resp> {
  // Receiver owned by "Actor" (jobs task)
  tx: mpsc::Sender<Command<Req, Resp>>,
}

impl Clone for JobDispatcher<Req, Resp> { ... }

impl JobDispatcher<Req, Resp> {
    pub async fn dispatch_job(&self, data: Req) -> oneshot::Receiver<Resp> {
      // Create command from data
      // Send command to jobs task
      // Return receiver from which job result can be collected
    }
}
    \end{lstlisting}
  \end{mdframed}
\end{frame}

\begin{frame}{To-Do List}
  \emph{server} crate:
  \begin{itemize}[<+->]
    \item<1> Implement unified and ergonomic way to handle shutdown signals
    \item<2> Add organised logging (with support for various levels of verbosity)
    \item<3> Consider compatability of serialized data across architectures, versions of this crate, versions
             of dependencies (look into protobuf). (May be difficult because input/output data is generic by design.)
    \item<4> Think carefully about backpressure
    \item<5> Ensure there are no partial-read vulnerabilities
    \item<6> Investigate automatically figuring out channel bounds, rather than asking the user to supply them
    \item<7> Automated testing
  \end{itemize}
\end{frame}

\begin{frame}{To-Do List}
  \emph{client} crate:
  \begin{itemize}[<+->]
    \item<1> Not I/O bound - does it need async? Would it be enough to have a read thread and a write thread?
  \end{itemize}
\end{frame}

\end{document}

